%%%%%%%%%%%%%%%%%%%%%%%%%%%%%%%%%%%%%%%%%
% Medium Length Professional CV
% LaTeX Template
% Version 2.0 (8/5/13)
%
% This template has been downloaded from:
% http://www.LaTeXTemplates.com
%
% Original author:
% Trey Hunner (http://www.treyhunner.com/)
%
% Important note:
% This template requires the resume.cls file to be in the same directory as the
% .tex file. The resume.cls file provides the resume style used for structuring the
% document.
%
%%%%%%%%%%%%%%%%%%%%%%%%%%%%%%%%%%%%%%%%%

%----------------------------------------------------------------------------------------
%	PACKAGES AND OTHER DOCUMENT CONFIGURATIONS
%----------------------------------------------------------------------------------------

\documentclass[a4paper]{resume} % Use the custom resume.cls style

\usepackage[left=0.75in,top=0.6in,right=0.75in,bottom=0.6in]{geometry} % Document margins
\usepackage{url}
\renewcommand{\labelenumi}{[\arabic{enumi}]}
\usepackage{etaremune}

\name{Publications} % Your name
\address{Atsuhisa Ota  (\url{a.ota(AT)damtp.cam.ac.uk})} % Your phone number and email
\address{\textit{Department of Applied Mathematics and Theoretical Physics, University of Cambridge}} % Your address
\address{\today}
\begin{document}




%----------------------------------------------------------------------------------------
%	EDUCATION SECTION
%----------------------------------------------------------------------------------------
\paragraph{Note:}
In my field~(High energy physics and Cosmology), relevant achievements are papers published in journals. 
Impact factors are similar for typical journals such as Physical Review D~(PRD), Physics Letters B~(PLB), Journal of Cosmology and Astroparticle Physics~(JCAP) and Journal of High Energy Physics~(JHEP), except for Physical Review Letter~(PRL). 
I have contributed actively to all of the publications below. 
The order of authors is strictly alphabetical {\bf except for {\bf A.~Ota} and N.~Bartolo (2018)}.
I am familiar with all the results presented there, and I know how to reproduce them. A paper posted on the arXiv is commonly considered published, so I add my latest works (arXiv:1808.10517 and arXiv:1805.06240), although they are still under peer review by PRL and JHEP.

\begin{rSection}{Papers in publication}


\begin{etaremune}


%\cite{Ota:2018iso}
\item
  {\bf A.~Ota},
  ``Statistical anisotropy in CMB spectral distortions,''
  arXiv:1810.03928 [astro-ph.CO], {\bf accepted by PLB}.
  %%CITATION = ARXIV:1810.03928;%%

\item
%\cite{Ota:2018zwm}
%\bibitem{Ota:2018zwm} 
  {\bf A.~Ota} and N.~Bartolo,
  ``CMB spectroscopy at third-order in cosmological perturbations,''\\
  arXiv:1808.10517 [astro-ph.CO], {\bf submitted to PRL}.
  %%CITATION = ARXIV:1808.10517;%%
    \item 
      M.~Hongo, S.~Kim, T.~Noumi and {\bf A.~Ota},
  ``Effective field theory of time-translational symmetry breaking in nonequilibrium open system,''
  arXiv:1805.06240 [hep-th], {\bf submitted to JHEP}.
  %%CITATION = ARXIV:1805.06240;%%


\item
  T.~Haga, K.~Inomata, {\bf A.~Ota} and A.~Ravenni,
  ``Exploring compensated isocurvature perturbations with CMB spectral distortion anisotropies,''
  JCAP {\bf 1808}, no. 08, 036 (2018)
  doi:10.1088/1475-7516/2018/08/036
  [arXiv:1805.08773 [astro-ph.CO]].
  %%CITATION = doi:10.1088/1475-7516/2018/08/036;%%
  2 citations counted in INSPIRE as of 28 Aug 2018
  
    \item 
  {\bf A.~Ota} and M.~Yamaguchi,
  ``Secondary isocurvature perturbations from acoustic reheating,''
  JCAP {\bf 1806}, no. 06, 022 (2018)
  doi:10.1088/1475-7516/2018/06/022
  [arXiv:1705.05196 [astro-ph.CO]].
  %%CITATION = doi:10.1088/1475-7516/2018/06/022;%%
  
    \item
  {\bf A.~Ota},
  ``CMB spectral distortions as solutions to the Boltzmann equations,''
  JCAP {\bf 1701}, no. 01, 037 (2017)
  doi:10.1088/1475-7516/2017/01/037
  [arXiv:1611.08058 [astro-ph.CO]].
  %%CITATION = doi:10.1088/1475-7516/2017/01/037;%%
  3 citations counted in INSPIRE as of 19 Aug 2018
  
    \item
  {\bf A.~Ota},
  ``Cosmological constraints from $\mu E$ cross-correlations,''
  Phys.\ Rev.\ D {\bf 94}, no. 10, 103520 (2016)
  doi:10.1103/PhysRevD.94.103520
  [arXiv:1607.00212 [astro-ph.CO]].
  %%CITATION = doi:10.1103/PhysRevD.94.103520;%%
  7 citations counted in INSPIRE as of 19 Aug 2018

  \item
    A.~Naruko, {\bf A.~Ota} and M.~Yamaguchi,
  ``Probing small-scale non-Gaussianity from anisotropies in acoustic reheating,''
  JCAP {\bf 1505}, no. 05, 049 (2015)
  doi:10.1088/1475-7516/2015/05/049
  [arXiv:1503.03722 [astro-ph.CO]].
  %%CITATION = doi:10.1088/1475-7516/2015/05/049;%%
  6 citations counted in INSPIRE as of 19 Aug 2018
  
    \item
  {\bf A.~Ota}, T.~Sekiguchi, Y.~Tada and S.~Yokoyama,
  ``Anisotropic CMB distortions from non-Gaussian isocurvature perturbations,''\\
  JCAP {\bf 1503}, no. 03, 013 (2015)
  doi:10.1088/1475-7516/2015/03/013
  [arXiv:1412.4517 [astro-ph.CO]].
  %%CITATION = doi:10.1088/1475-7516/2015/03/013;%%
  9 citations counted in INSPIRE as of 19 Aug 2018
  
    \item 
      {\bf A.~Ota}, T.~Takahashi, H.~Tashiro and M.~Yamaguchi,
  ``CMB $\mu$ distortion from primordial gravitational waves,''
  JCAP {\bf 1410}, no. 10, 029 (2014)
  doi:10.1088/1475-7516/2014/10/029
  [arXiv:1406.0451 [astro-ph.CO]].
  %%CITATION = doi:10.1088/1475-7516/2014/10/029;%%
  14 citations counted in INSPIRE as of 19 Aug 2018
    
\end{etaremune}

\newpage
\end{rSection}

\begin{rSection}{presentations}

\noindent\textit{Invited talks}
\begin{etaremune}
	 \item
``CMB spectroscopy for primordial non-Gaussianity'', Probing fundamental physics with CMB spectral distortions, CERN, Switzerland, 03/2018
    \item 
``CMB $\mu$ distortion from primordial gravitational waves'', Mini-Workshop on Cosmology, Asia Pacific Center of Theoretical Physics, Korea, 12/2014
\end{etaremune}

\noindent\textit{Conference talks~(International)}

\begin{etaremune}
	\item
    ``Effective field theory of time-translational symmetry breaking in open systems'', Infrared physics of gauge theories and quantum dynamics of inflation, Shiga, Japan, 01/2018
    
    \item
    ``Spontaneous symmetry breaking in open systems: Toward application to EFT of inflation'', The 27th Workshop on General Relativity and Gravitation, Hiroshima, Japan 11/2017
    \item
    ``Cosmological constraints from $\mu$E cross-correlations'', The 26th Workshop on General Relativity and Gravitation, Osaka City University, Japan, 10/2016
    \item
    ``Cosmological constraints from $\mu$E cross-correlations'', Summer School on Symmetries, Fundamental Interactions and Cosmology 2016, Germany, 09/2016
    \item
    ``Cosmological constraints from $\mu$E cross-correlations'', RESCUE Summer school, Nagano, Japan, 08/2016
    \item
``CMB $\mu$ distortion from primordial gravitational waves'' The 24th Workshop on General Relativity and Gravitation, Kavli Institute for Physics and Mathematics of the Universe, Tokyo, Japan 11/2014     
    \item 
    ``CMB $\mu$ distortion from primordial gravitational waves'' Research Center of the Early Universe Summer school, Nagano Japan, 08/2014
    
        
\end{etaremune}

\noindent\textit{Conference talks~(in Japanese)}

\begin{etaremune}
    \item 
    ``Secondary isocurvature perturbations from acoustic reheating'', JPS conference, 09/2017
    \item
    ``CMB spectral distortions as solutions of the Boltzmann equation'', JPS conference, 03/2017
    \item
    Other three JPS presentations, but titles are in Japanese.
\end{etaremune}


\noindent\textit{Seminar talks}

\begin{etaremune}
    \item
    ``Statistical anisotropy in spectral distortions'', University of Sussex, 12/2018
    \item
    ``CMB spectral distortions in the framework of cosmological perturbation theory'', Manchester University, 11/2018
    
    \item
    ``Statistical anisotropy in spectral distortions'', ICG Portsmouth, 10/2018

    \item
    ``EFT of dissipative fluid: an application to cosmology'', The Institut d'Astrophysique de Paris (IAP), 10/2018

    \item
    ``EFT of dissipative fluid: an application to cosmology'', Marseille University, 10/2018

    \item
    ``Late time evolution of isocurvature perturbations at second order'', Padova University, 05/2018

    \item
    ``Late time evolution of isocurvature perturbations at second order'', Groningen University, 05/2018

    \item
    ``Open system effective field theory for time-translational symmetry breaking: towards application to EFT of inflation'', Utrecht University, 04/2018

    \item
    ``A new method for higher order CMB anisotropy'', Rikkyo University, 05/2017

    \item
    ``A new method for higher order CMB anisotropy'', Kobe University, 03/2017  

    \item
    ''CMB spectral distortions as solutions to the Boltzmann equations'', KEK, 03/2017  

    \item
    ``Higher order anisotropies in the cosmic microwave background'', Max Planck Institute of Astrophysics (Job Talk), 01/2017

    \item
    ``Cosmological Constraints from $\mu$E cross-correlations'' Syracuse University, 10/2016

    \item
    ``Cosmological Constraints from $\mu$E cross-correlations'' Manchester University, 10/2016 

    \item
    ``Cosmological Constraints from $\mu$E cross-correlations'' Utrecht University, 09/2016 

    \item
``Cosmological Constraints from $\mu$E cross-correlations'' Max Planck Institute of Astrophysics, 09/2016 

    \item
``The cosmic microwave background $\mu$-distortions and primordial gravitational waves'', University of Nagoya, 09/2014 

    \item 
    ``CMB $\mu$ distortion and primordial gravitational waves'' Joint Seminar, Tokyo Institute of Technology, 05/2014
\end{etaremune}

\end{rSection}


\begin{rSection}{Prize and Awards}
	\begin{etaremune}

\item Overseas Research Fellow of the Japan Society for the Promotion of Science, 01/04/2018 - 31/03/2020

    \item Research Fellow of the Japan Society for the Promotion of Science (PD), 01/04/2017 - 31/03/2018

    \item Research Fellow of the Japan Society for the Promotion of Science (DC2), 01/04/2016 - 31/03/2017

    \item
Vice valedictorian, Department of Physics, Tokyo Institute of Technology, 03/2012
    
\end{etaremune}
\end{rSection}

\begin{rSection}{Fundings}
\begin{etaremune}
    \item
Grant-in-Aid for JSPS Fellows, Project/Area Number: 16J03220, Project Period (FY)    22/04/2016 - 31/02/2018,
Fiscal Year 2017 : 1,430,000JPY (Direct Cost : 1,100,000JPY, Indirect Cost : 330,000JPY)
Fiscal Year 2016 : 1,200,000JPY (Direct Cost : 1,200,000JPY)

\end{etaremune}
\end{rSection}

\begin{rSection}{Supervising and mentoring activities}
	\begin{etaremune}
    \item (Unofficial) supervision of a Master student, Tokyo Institute of Technology, 2017 - 2018
    \item Teaching assistant, Tokyo Institute of Technology, Mechanics, Electromagnetism, Analytical Mechanics, 10/2011 - 03/2017
\end{etaremune}
\end{rSection}




\end{document}
